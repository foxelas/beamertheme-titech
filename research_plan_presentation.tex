\documentclass{beamer}
\usetheme{titech}

\usepackage{enumitem}
\setitemize{label=\usebeamerfont*{itemize item}%
  \usebeamercolor[fg]{itemize item}
  \usebeamertemplate{itemize item}}
  
\graphicspath{{img/}, {add_img/}} %Setting the graphicspath 
\usepackage{amsmath}
\usepackage{mathtools}
\newcommand\norm[1]{\left\lVert#1\right\rVert}
\usepackage[square, sort&compress, numbers]{natbib}
\usepackage{wrapfig}
\usepackage{hyperref}
\usepackage{booktabs}
\usepackage{subcaption}
\usepackage{comment}

\usepackage{tikz}
\usetikzlibrary{arrows}
\usepackage{datenumber}
\usepackage{xifthen}
\usepackage{ifthen}
% counters for calculating with dates
\newcounter{startdate}
\newcounter{enddate}
\newcounter{tempdate}
\newcounter{dateone}
\newcounter{datetwo}

% 
\newcommand{\startenddiff}[6]{%
\setmydatenumber{startdate}{#1}{#2}{#3}%
\setmydatenumber{enddate}{#4}{#5}{#6}%
\setmydatenumber{tempdate}{#4}{#5}{#6}%
\addtocounter{tempdate}{-\thestartdate}%
}

%
\newcommand{\datediff}[6]%
{   \setmydatenumber{dateone}{#1}{#2}{#3}
    \setmydatenumber{datetwo}{#4}{#5}{#6}
    \addtocounter{dateone}{-\thestartdate}
    \addtocounter{datetwo}{-\thestartdate}
}

%
\newcommand{\drawtimeline}%
{   \setdatebynumber{\thestartdate}
    \pgfmathtruncatemacro{\numberofdays}{\thetempdate}
    \pgfmathsetmacro{\daywidth}{\timelinewidth/\numberofdays}
    \draw[-stealth] (0,0) -- (\timelinewidth,0) -- ++(0.3,0);
    \foreach \x in {0,...,\numberofdays}
    { \ifthenelse{\thedateday = 1}
        { \ifcase\thedatemonth
            \or \xdef\monthname{Jan}
            \or \xdef\monthname{Feb}
            \or \xdef\monthname{Mar}
            \or \xdef\monthname{Apr}
            \or \xdef\monthname{May}
            \or \xdef\monthname{Jun}
            \or \xdef\monthname{Jul}
            \or \xdef\monthname{Aug}
            \or \xdef\monthname{Sep}
            \or \xdef\monthname{Oct}
            \or \xdef\monthname{Nov}
            \or \xdef\monthname{Dec}    
            \else       
            \fi
            \draw (\x*\daywidth,0) -- (\x*\daywidth,0.25) node[right,rotate=90,font=\tiny]{}; %{\monthname\ \thedateyear};%
        }{}
        \ifthenelse{\equal{\datedayname}{Monday}}
        { \draw (\x*\daywidth,0) -- (\x*\daywidth,-0.05);
        }{}
        \addtocounter{datenumber}{1}
        \setdatebynumber{\thedatenumber}
    }
}

\newcommand{\timeentry}[8][gray]% [options] start date, end date, description
{ \datediff{#2}{#3}{#4}{#5}{#6}{#7}
    \pgfmathtruncatemacro{\numberofdays}{\thetempdate}
    \pgfmathsetmacro{\daywidth}{\timelinewidth/\numberofdays}
    \draw[opacity=1,line width=1.5mm,line cap=round,#1] (\thedateone*\daywidth,0) -- (\thedatetwo*\daywidth,0) node[left,rotate=60,pos=0.5] {#8};
}


%----------------------------------------------------------------------------------------
%	 TITLE SLIDE
%----------------------------------------------------------------------------------------

\title{Title \\ Title}
\subtitle{ \textcolor{mDarkYellow} {Research Planning}}
\author{Aloupogianni Eleni}
\institute{Tokyo Institute of Technology | Obi Laboratory}
\date{{\scriptsize \today}}


%------------------------------------------------

\begin{document}

%------------------------------------------------

\begin{frame}[noframenumbering,plain]
	\maketitle % Automatically created using the information in the commands above
\end{frame}

%----------------------------------------------------------------------------------------
%	 Contents 
%----------------------------------------------------------------------------------------
\begin{frame}{Contents}
   \tableofcontents
\end{frame}

%----------------------------------------------------------------------------------------
%	 SECTION 1
%----------------------------------------------------------------------------------------

{\AtBeginSection{}
\section{Main} % Section title slide, unnumbered
}
\begin{frame}{Background}

\begin{tabular}{cl}  
\begin{tabular}{c}
%\begin{figure}
\centering \includegraphics[width=0.3\textwidth]{example-image-a}
%\caption{Example of positive margin determination}\label{fig:1}
%\end{figure}
\end{tabular}
& \begin{tabular}{l}
\parbox{0.5\linewidth}{%  change the parbox width as appropiate
 \textbf{a}: aa 
 \begin{itemize}
 \item Example of positive margin determination
 \item Example of positive margin determination
 \item Example of positive margin determination
 \end{itemize}
    }
\end{tabular}  \\
\end{tabular}


\begin{exampleblock}{s}
asdasdasd
\end{exampleblock}

as \textcolor{mDarkYellowComp}{as } retert

\end{frame}

%------------------------------------------------

\begin{frame}{Hypothesis \& Purpose}

\begin{itemize}
\item Example of positive margin determination \textcolor{mDarkYellowComp}{spectral absorbance and reflectance}
\item SExample of positive margin determination \textcolor{mDarkYellowComp}{skin chromophores}. 
 
\end{itemize}


\begin{alertblock}{Hypothesis}
Example of positive margin determination
\end{alertblock}

\begin{exampleblock}{Purpose}
Example of positive margin determination
\end{exampleblock}

\end{frame}

%------------------------------------------------

\begin{frame}{Materials \& Methods}
   
  
\end{frame}

%------------------------------------------------

\begin{frame}{Current Progress}



\begin{figure}
\centering
\begin{subfigure}{0.3\textwidth}
  \centering
	\includegraphics[width=\textwidth]{example-image-a}
	%\caption{Melanin distribution map.}
\end{subfigure} %
\begin{subfigure}{0.3\textwidth}
  \centering
	\includegraphics[width=\textwidth]{example-image-a}
	%\caption{Close-up of hemoglobin distribution map.}
\end{subfigure}
\caption{ssdsdsd.}
\end{figure}


\end{frame}

%------------------------------------------------

\begin{frame}{Research Schedule}

% ===== user's choices =========================
\startenddiff{2020}{09}{29}{2022}{09}{29}
\pgfmathsetmacro{\timelinewidth}{10}% will be 0.3cm wider for arrow tip
% ==============================================
\begin{tikzpicture}
    \drawtimeline

    \timeentry[mDarkYellow]{2020}{10}{28}{2020}{11}{05}{Oct'20}
    \timeentry[mDarkYellow]{2021}{02}{10}{2021}{02}{20}{Feb'21}
    \timeentry[mDarkYellow]{2021}{08}{10}{2021}{08}{20}{Aug'21}
    \timeentry[mDarkYellow]{2021}{11}{10}{2021}{11}{20}{Nov'21}
    \timeentry[mDarkYellow]{2021}{12}{10}{2021}{12}{20}{Dec'21}
    \timeentry[mDarkYellow]{2022}{03}{10}{2022}{03}{20}{Mar'22}
    \timeentry[mDarkYellow]{2022}{04}{10}{2022}{04}{20}{Apr'22}
    \timeentry[mDarkYellow]{2022}{06}{10}{2022}{06}{20}{Jun'22}

\end{tikzpicture}

\begin{itemize}
\item<1> October 2020: Test run experimental set-up 
\item<1> February 2021: Create base analysis process 
\item<1> August 2021: Finish collecting first batch of image data
\item<1> November 2021: Calibrate algorithmic prototype
\item<1> December 2021: Adjust with feedback from doctors
\item<1> March 2022: Finish collecting second batch of image data
\item<1> April 2022: Adjust to complete dataset
\item<1> June 2022: Compile dissertation thesis 
\end{itemize}

\end{frame}

\section{Q \& A}
%------------------------------------------------

\begin{frame}{Equipment}
\href{https://www.erere.co.jp/}{link1}

\end{frame}

%------------------------------------------------

\begin{frame}
	Thank you for your attention!
    \newline
    \textcolor{mDarkYellow}{Any questions?}
\end{frame}

%------------------------------------------------

% \appendix

\begin{frame}[allowframebreaks]{Bibliography}
\begin{tiny}
   \nocite{*}
    \bibliography{bibliography}
    \bibliographystyle{unsrtnat}
\end{tiny}
\end{frame}


\end{document}
